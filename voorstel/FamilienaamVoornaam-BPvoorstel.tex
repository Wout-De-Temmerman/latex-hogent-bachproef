%==============================================================================
% Sjabloon onderzoeksvoorstel bachproef
%==============================================================================
% Gebaseerd op document class `hogent-article'
% zie <https://github.com/HoGentTIN/latex-hogent-article>

% Voor een voorstel in het Engels: voeg de documentclass-optie [english] toe.
% Let op: kan enkel na toestemming van de bachelorproefcoördinator!
\documentclass{hogent-article}

% Invoegen bibliografiebestand
\addbibresource{voorstel.bib}

% Informatie over de opleiding, het vak en soort opdracht
\studyprogramme{Professionele bachelor toegepaste informatica}
\course{Bachelorproef}
\assignmenttype{Onderzoeksvoorstel}
% Voor een voorstel in het Engels, haal de volgende 3 regels uit commentaar
% \studyprogramme{Bachelor of applied information technology}
% \course{Bachelor thesis}
% \assignmenttype{Research proposal}

\academicyear{2024-2025} % TODO: pas het academiejaar aan

% TODO: Werktitel
\title{Automatisatie van de transitie naar IPv6-netwerken voor studentenkoten en onderwijsinstellingen binnen Eduroam}

% TODO: Studentnaam en emailadres invullen
\author{Wout De Temmerman}
\email{wout.detemmerman@student.hogent.be}

% TODO: Geef de co-promotor op
\supervisor[Co-promotor]{N. Verslycken (onder voorbehoud) (SmartEye, \href{mailto:nicky.verslycken@smarteye.eu}{nicky.verslycken@smarteye.eu})}

% Binnen welke specialisatierichting uit 3TI situeert dit onderzoek zich?
% Kies uit deze lijst:
%
% - Mobile \& Enterprise development
% - AI \& Data Engineering
% - Functional \& Business Analysis
% - System \& Network Administrator
% - Mainframe Expert
% - Als het onderzoek niet past binnen een van deze domeinen specifieer je deze
%   zelf
%
\specialisation{System \& Network Administrator}
\keywords{IPv6, Dual-Stack, Netwerken, Eduroam}

\begin{document}

\begin{abstract}
  SmartEye beheert meer dan 700 netwerklocaties, zowel voor particuliere kotuitbaters als voor studentenvoorzieningen zoals Upkot, Xior en zelfs voor onderwijsinstellingen zoals HoGent en Artevelde.
  Deze diversiteit aan infrastructuur maakt het uitrollen van netwerkupgrades uitdagend, waardoor er gebruik gemaakt moet worden van automatiseringstools zoals Ansible.
  De volgende stap in de toekomst van SmartEye is de overgang naar IPv6, een upgrade aan hun netwerk die al hun locaties beter futureproof maakt.
  De gekozen methode hiervoor zal afhangen van meerdere factoren. De implementatie van een dual-stack netwerk, waarbij zowel IPv4 als IPv6 worden ondersteund, wordt beschouwd als de meest robuuste en door SmartEye zelf verkozen aanpak.
  Toch moet worden getest of dit haalbaar is, aangezien er gebruik wordt gemaakt van een groot assortiment aan toestellen die mogelijk niet allemaal compatibel zijn met deze technologie.
  Dit onderzoek richt zich op het ontwikkelen van een geautomatiseerde methode om deze transitie uit te voeren voor de verschillende types en groottes van infrastructuur.
\end{abstract}

\tableofcontents

% De hoofdtekst van het voorstel zit in een apart bestand, zodat het makkelijk
% kan opgenomen worden in de bijlagen van de bachelorproef zelf.
%---------- Inleiding ---------------------------------------------------------

% TODO: Is dit voorstel gebaseerd op een paper van Research Methods die je
% vorig jaar hebt ingediend? Heb je daarbij eventueel samengewerkt met een
% andere student?
% Zo ja, haal dan de tekst hieronder uit commentaar en pas aan.

\paragraph{Opmerking}

Dit voorstel is gebaseerd op het onderzoeksvoorstel dat werd geschreven in het
kader van het vak Research Methods dat ik vorig academiejaar heb
uitgewerkt met \\ Maarten Adriaenssens als mede-auteur.


\section{Inleiding}%
\label{sec:inleiding}

Een netwerk kan snel futureproof worden gemaakt door de implementatie van IPv6, een relatief nieuwe technologie die het probleem van de gelimiteerde 32-bit adressen in IPv4 oplost
door gebruik te maken van 128-bit adressen. Hoewel deze technologie al sinds 1998 bestaat, blijft de implementatie ervan beperkt. 
Dit blijkt duidelijk uit de IPv6 Support Rate-grafiek zoals gezien op de website van \textcite{EuropeanCommission}, 
waaruit blijkt dat in België aan het begin van 2024 slechts 59\% van de eindgebruikers toegang heeft tot de technologie, 
terwijl slechts 29\% van de services compatibel is met IPv6.

SmartEye wil ervoor zorgen dat het Eduroam-netwerk dat ze hebben opgezet en onderhouden, de nodige upgrade krijgt, 
en zoekt daarvoor een geautomatiseerde oplossing. Een eerste probleem hierbij is dat de netwerken die zijn uitgebouwd, 
niet allemaal op dezelfde manier functioneren. De netwerkapparatuur die is geïnstalleerd bestaat uit onder andere Cisco- en MikroTik-apparatuur, 
waarbij elk toestel op een andere manier moet worden geconfigureerd.  
De configuratie van een MikroTik-toestel vereist eerst een certificaat, dat door SmartEye zelf wordt voorzien voor de uitwerking van deze proef. 
Een dual-stack netwerk heeft de voorkeur, aangezien dit geïmplementeerd kan worden zonder het risico dat bestaande opstellingen niet meer werken. 
Daarnaast zal NAT64 aan bod komen om toestellen die IPv6 niet ondersteunen toch te laten communiceren met het netwerk. Dit alles draagt bij aan de robuustheid van het netwerk, 
aangezien beide technologieën dan op alle locaties gebruikt kunnen worden. \\

Het doel van de proef is om een algemeen script te ontwikkelen dat gebruikt kan worden om alle netwerktoestellen binnen het Eduroam-netwerk voor te bereiden op het gebruik van het nieuw verkregen IPv6-adres. 
Dit script zal niet alleen op de externe toestellen worden toegepast, maar ook op de backbone van SmartEye zelf. 
De ontwikkeling van het script zal plaatsvinden in een labo-omgeving die door SmartEye is voorzien.

%---------- Stand van zaken ---------------------------------------------------

\section{Literatuurstudie}%
\label{sec:literatuurstudie}

\subsection{Implementatie}
\label{sec:implementatie}

Zoals eerder vermeld is de implementatie van IPv6 beperkt. 
Dit terwijl marktleiders van internet service providers zoals Cloudflare, Proximus en Telenet dicht tegen de 100\% zitten op het vlak van aangeboden IPv6 percentage \autocite{Test2024}. 
Ook op de website van \textcite{EuropeanCommission} is zichtbaar dat op vlak van end-user implementatie België deel van de marktleiders is, terwijl de services nog geen 30\% haalt.

\subsection{Netwerk-configuratie via Ansible}
\label{sec:netconfigAnsible}

Een geautomatiseerde uitrol van netwerkconfiguraties kan worden uitgevoerd met Ansible, waarbij onderscheid moet worden gemaakt tussen de verschillende merken netwerkapparatuur. 
Hiervoor zal de configuratie worden opgesplitst in verschillende YAML-bestanden, die in het Ansible-playbook worden aangeroepen. 
Deze YAML-bestanden bevatten ten minste de volgende componenten: 
functies voor het instellen van IPv6-adressen op de interfaces, 
de juiste configuratie van de loopback-interface, en de minimale configuratie voor EIGRP of OSPF \autocite{MohdFuziMohdFaris2021}.

\subsection{IPv4 vs IPv6}
\label{sec:IPv4vsIPv6}

Met de huidige snelheid waarmee het internet groeit en evolueert, wordt het steeds duidelijker dat de huidige toewijzing van IP-adressen niet zal kunnen bijhouden. 
Zelfs met technologieën zoals CIDR1 en NAT2 zal dit probleem niet worden opgelost, alleen uitgesteld \autocite{Gu2005}.
Gelukkig bestaat er al een andere oplossing: het gebruik van IPv6, aangezien dit 128 bits gebruikt voor zijn adresseringsformaat in plaats van de 32 bits van IPv4. 
De huidige netwerkinfrastructuren bieden verschillende manieren om IPv6 te implementeren: dual-stack, tunneling of vertaling \autocite{Zunainah2021}. 
Deze implementatiemethoden hebben elk hun eigen voordelen en nadelen. De keuze is sterk afhankelijk van de huidige netwerktopologie en -infrastructuur. 
Hoewel op het eerste gezicht een dual-stack de oplossing lijkt, moet er verder onderzoek worden gedaan om de optimale keuze te bepalen \autocite{Zehl2001}.

% Voor literatuurverwijzingen zijn er twee belangrijke commando's:
% \autocite{KEY} => (Auteur, jaartal) Gebruik dit als de naam van de auteur
%   geen onderdeel is van de zin.
% \textcite{KEY} => Auteur (jaartal)  Gebruik dit als de auteursnaam wel een
%   functie heeft in de zin (bv. ``Uit onderzoek door Doll & Hill (1954) bleek
%   ...'')

%---------- Methodologie ------------------------------------------------------
\section{Methodologie}%
\label{sec:methodologie}


%---------- Verwachte resultaten ----------------------------------------------
\section{Verwacht resultaat, conclusie}%
\label{sec:verwachte_resultaten}

Hier beschrijf je welke resultaten je verwacht. Als je metingen en simulaties uitvoert, kan je hier al mock-ups maken van de grafieken samen met de verwachte conclusies. Benoem zeker al je assen en de onderdelen van de grafiek die je gaat gebruiken. Dit zorgt ervoor dat je concreet weet welk soort data je moet verzamelen en hoe je die moet meten.

Wat heeft de doelgroep van je onderzoek aan het resultaat? Op welke manier zorgt jouw bachelorproef voor een meerwaarde?

Hier beschrijf je wat je verwacht uit je onderzoek, met de motivatie waarom. Het is \textbf{niet} erg indien uit je onderzoek andere resultaten en conclusies vloeien dan dat je hier beschrijft: het is dan juist interessant om te onderzoeken waarom jouw hypothesen niet overeenkomen met de resultaten.



\printbibliography[heading=bibintoc]

\end{document}