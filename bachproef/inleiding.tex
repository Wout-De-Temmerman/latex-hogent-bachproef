%%=============================================================================
%% Inleiding
%%=============================================================================

\chapter{\IfLanguageName{dutch}{Inleiding}{Introduction}}%
\label{ch:inleiding}

\section{\IfLanguageName{dutch}{Probleemstelling}{Problem Statement}}%
\label{sec:probleemstelling}

Door de implementatie van IPv6 kan een netwerk snel futureproof worden gemaakt. 
Het is een relatief nieuwe technologie die het probleem van de gelimiteerde 32-bit adressen in IPv4 oplost door gebruik te maken van 128-bit adressen. 
Hoewel IPv6 al sinds 1998 bestaat, blijft de toepassing ervan beperkt. Dit blijkt duidelijk uit de IPv6 Support Rate-grafiek op de website van \textcite{EuropeanCommission}, 
waaruit blijkt dat in België begin 2024 59\% van de eindgebruikers toegang heeft tot IPv6, terwijl slechts 29\% van de services compatibel zijn met de technologie.
\\

Deze cijfers tonen aan dat de adoptie van IPv6 groeit aan constante snelheid, maar dat er nog steeds een duidelijke kloof bestaat tussen netwerktoegang en implementatie. 
Aangezien bedrijven en organisaties steeds vaker overschakelen op IPv6, wordt het voor toekomstige IT-professionals essentieel om hier grondige kennis over te verwerven. 
Daarom is het belangrijk dat de opleiding IPv6 een plaats geeft binnen het curriculum, zodat studenten goed voorbereid zijn op de netwerkinfrastructuur van de toekomst.
\\

Het netwerklabo op campus Schoonmeersen van HOGENT wordt frequent gebruikt voor demonstraties en proefopstellingen in de lessen Netwerken 1 tot 4. 
Echter, het labo mist een essentieel component: een IPv6-verbinding. 
Een mogelijke oplossing is het gebruik van een node op de servers van IDLab aan de UGent. 
Via deze node kan een tunnel worden opgezet, waardoor het netwerkverkeer via het UGent-netwerk – dat wél IPv6 ondersteunt – kan worden gerouteerd.
\\

Deze opstelling kan al worden gerealiseerd met een bestaand script, geschreven door dhr. Van Maele. 
Dit script legt een verbinding tussen een virtuele machine en de node in IDLab, waarbij de node fungeert als OpenVPN-server en de virtuele machine als OpenVPN-client. 
Daarnaast wordt er een Layer 2-tunnel opgezet tussen deze twee punten, waarop de OpenVPN-verbinding steunt.
Het probleem hierbij is echter dat het opzetten van deze configuratie veel tijd en moeite vergt van de lectoren, 
of zelfs niet praktisch uitvoerbaar is vanwege het gebrek aan gebruiksvriendelijkheid van het script.

\section{\IfLanguageName{dutch}{Onderzoeksvraag}{Research question}}%
\label{sec:onderzoeksvraag}

De vraag waarover dit onderzoek een oplossing probeert te vinden gaat als volgt:
Kan de opzet van een IPv6 connectie aan de hand van een tunnel geautomatiseerd worden met Ansible?

Om deze verder onder te verdelen zal er vooral onderzoek worden gedaan naar volgende vragen:
\begin{itemize}
    \item Waar liggen de essentiële verschillen tussen Layer 2 en OpenVPN tunnels?
    \item Is er een vershil op vlak van performantie?
    \item Is het mogelijk om een connectie tussen 2 netwerken op te zetten met enkel een Layer 2 tunnel?
    \item Kan het bestaande script in een Ansible Playbook worden verwerkt?
    \item Wat zijn de resultaten na het testen in effectieve omgeving?
\end{itemize}

\section{\IfLanguageName{dutch}{Onderzoeksdoelstelling}{Research objective}}%
\label{sec:onderzoeksdoelstelling}

Dit onderzoek richt zich op het in kaart brengen van de verschillen tussen deze tunnelingmethodes en het testen van mogelijke alternatieve methoden. 
Uiteindelijk wordt de meest efficiënte en correcte implementatie verwerkt in een Ansible Playbook, dat automatisch de connectie tussen beide toestellen opzet.
Deze oplossing kan dan worden toegepast bij toekomstige labo’s in het netwerklabo, zodat snel en eenvoudig een functionerende IPv6-verbinding kan worden opgezet. 
Hierdoor kunnen studenten niet alleen de theorie over IPv6 bestuderen, maar deze ook daadwerkelijk testen in een realistische omgeving.

\section{\IfLanguageName{dutch}{Opzet van deze bachelorproef}{Structure of this bachelor thesis}}%
\label{sec:opzet-bachelorproef}

% Het is gebruikelijk aan het einde van de inleiding een overzicht te
% geven van de opbouw van de rest van de tekst. Deze sectie bevat al een aanzet
% die je kan aanvullen/aanpassen in functie van je eigen tekst.

De rest van deze bachelorproef is als volgt opgebouwd:

In Hoofdstuk~\ref{ch:stand-van-zaken} wordt een overzicht gegeven van de stand van zaken binnen het onderzoeksdomein, op basis van een literatuurstudie.

In Hoofdstuk~\ref{ch:methodologie} wordt de methodologie toegelicht en worden de gebruikte onderzoekstechnieken besproken om een antwoord te kunnen formuleren op de onderzoeksvragen.

% TODO: Vul hier aan voor je eigen hoofstukken, één of twee zinnen per hoofdstuk

In Hoofdstuk~\ref{ch:conclusie}, tenslotte, wordt de conclusie gegeven en een antwoord geformuleerd op de onderzoeksvragen. Daarbij wordt ook een aanzet gegeven voor toekomstig onderzoek binnen dit domein.