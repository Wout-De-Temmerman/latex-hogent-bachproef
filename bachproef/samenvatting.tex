%%=============================================================================
%% Samenvatting
%%=============================================================================

% TODO: De "abstract" of samenvatting is een kernachtige (~ 1 blz. voor een
% thesis) synthese van het document.
%
% Een goede abstract biedt een kernachtig antwoord op volgende vragen:
%
% 1. Waarover gaat de bachelorproef?
% 2. Waarom heb je er over geschreven?
% 3. Hoe heb je het onderzoek uitgevoerd?
% 4. Wat waren de resultaten? Wat blijkt uit je onderzoek?
% 5. Wat betekenen je resultaten? Wat is de relevantie voor het werkveld?
%
% Daarom bestaat een abstract uit volgende componenten:
%
% - inleiding + kaderen thema
% - probleemstelling
% - (centrale) onderzoeksvraag
% - onderzoeksdoelstelling
% - methodologie
% - resultaten (beperk tot de belangrijkste, relevant voor de onderzoeksvraag)
% - conclusies, aanbevelingen, beperkingen
%
% LET OP! Een samenvatting is GEEN voorwoord!

%%---------- Nederlandse samenvatting -----------------------------------------
%
% TODO: Als je je bachelorproef in het Engels schrijft, moet je eerst een
% Nederlandse samenvatting invoegen. Haal daarvoor onderstaande code uit
% commentaar.
% Wie zijn bachelorproef in het Nederlands schrijft, kan dit negeren, de inhoud
% wordt niet in het document ingevoegd.

\IfLanguageName{english}{%
\selectlanguage{dutch}
\chapter*{Samenvatting}
\selectlanguage{english}
}{}

%%---------- Samenvatting -----------------------------------------------------
% De samenvatting in de hoofdtaal van het document

\chapter*{\IfLanguageName{dutch}{Samenvatting}{Abstract}}

% SmartEye beheert meer dan 700 netwerklocaties, waaronder particuliere kotuitbaters, studentenvoorzieningen zoals Upkot en Xior, en onderwijsinstellingen zoals HOGENT en Artevelde. 
% De diversiteit aan infrastructuur maakt het uitrollen van netwerkupgrades uitdagend, vooral omdat elk type netwerkapparatuur een verschillende configuratie vereist. 
% Met de geplande overgang naar IPv6 staat SmartEye voor de taak om hun netwerken toekomstbestendig te maken met een geautomatiseerde oplossing. 
% Hiervoor willen ze gebruik maken van Ansible Playbooks, maar er is nog onvoldoende duidelijkheid over wat er nodig is om een goed werkend playbook op te stellen dat aan al hun eisen voldoet. 
% Dit onderzoek richt zich op het scheppen van duidelijkheid over de noodzakelijke componenten bij het opstellen van een Ansible Playbook voor de ondersteuning van IPv6 in verschillende netwerkinfrastructuren. 
% De focus ligt op het in kaart brengen van vereisten en best practices, het identificeren van mogelijke knelpunten en het onderzoeken van netwerkapparaten die extra aandacht nodig hebben voor compatibiliteit, 
% met specifieke aandacht voor de configuratie van MikroTik-toestellen, aangezien deze de backbone van het netwerk vormen.
% \\

% Onderzoek werd gedaan met een combinatie van literatuurstudie en experimentele implementatie binnen het netwerk van SmartEye toegepast. 
% Dit omvatte een analyse van de MikroTik documentatie, het opzetten van een testomgeving, het ontwikkelen en testen van Ansible Playbooks en een evaluatie van de prestaties en betrouwbaarheid van de playbooks in een realistische netwerkomgeving. 
% Uit de resultaten blijkt dat een dual-stack implementatie de beste optie is om met IPv6 te starten, aangezien dit de downtime het laagst houdt. 
% Indien SmartEye dan achteraf wilt overschakelen naar een puur IPv6 netwerk kan dit ook gedaan worden zonder opmerkelijke downtime te creëren.
% \\

% De resultaten tonen aan dat automatisering met Ansible Playbooks een effectieve manier is om de overstap naar IPv6 te ondersteunen, mits er rekening wordt gehouden met compatibiliteitsproblemen en hardwarebeperkingen. 
% Verder onderzoek kan zich richten op bredere compatibiliteitskwesties met andere netwerkfabrikanten en de integratie van beveiligingsmaatregelen binnen de IPv6-configuratie.


TODO