%%=============================================================================
%% Methodologie
%%=============================================================================

\chapter{\IfLanguageName{dutch}{Methodologie}{Methodology}}%
\label{ch:methodologie}

%% TODO: In dit hoofstuk geef je een korte toelichting over hoe je te werk bent
%% gegaan. Verdeel je onderzoek in grote fasen, en licht in elke fase toe wat
%% de doelstelling was, welke deliverables daar uit gekomen zijn, en welke
%% onderzoeksmethoden je daarbij toegepast hebt. Verantwoord waarom je
%% op deze manier te werk gegaan bent.
%% 
%% Voorbeelden van zulke fasen zijn: literatuurstudie, opstellen van een
%% requirements-analyse, opstellen long-list (bij vergelijkende studie),
%% selectie van geschikte tools (bij vergelijkende studie, "short-list"),
%% opzetten testopstelling/PoC, uitvoeren testen en verzamelen
%% van resultaten, analyse van resultaten, ...
%%
%% !!!!! LET OP !!!!!
%%
%% Het is uitdrukkelijk NIET de bedoeling dat je het grootste deel van de corpus
%% van je bachelorproef in dit hoofstuk verwerkt! Dit hoofdstuk is eerder een
%% kort overzicht van je plan van aanpak.
%%
%% Maak voor elke fase (behalve het literatuuronderzoek) een NIEUW HOOFDSTUK aan
%% en geef het een gepaste titel.

\section{Voorbereiding}
\label{sec:voorbereiding}

Als eerste stap van het project wordt ervoor gezorgd dat de nodige machines correct zijn opgesteld. Dit omvat de volgende componenten:

\begin{itemize}
    \item \textbf{Een virtuele machine binnen het netwerklabo:}  
    Deze draait op een server en fungeert als een van de uiteinden van de tunnel. De machine zorgt voor IPv6-connectiviteit binnen het labo.  
    De rol van dit toestel is als volgt: het functioneert als een router die alle aangesloten toestellen in het netwerklabo voorziet van IPv6, zodat studenten hun proeven via het bekabelde netwerk kunnen uitvoeren.
    Deze verbinding wordt tot stand gebracht via een aangesloten switch, waaraan de studenten hun toestellen verbinden.

    \item \textbf{Een tunnelverbinding:}  
    De server zet een tunnel op, waarbij er gebruik wordt gemaakt van ofwel een pure Layer 2-verbinding, of een combinatie van Layer 2 en een VPN-verbinding. De data wordt via deze tunnel doorgestuurd naar een node binnen het IDLab.

    \item \textbf{De node binnen het IDLab:}  
    Deze heeft als enige taak al het verkeer te routeren naar een publiek IPv6-adres en fungeert zo als de default gateway voor de gebruikers binnen het netwerklabo.
\end{itemize}

\begin{figure}[H]
    \centering
    \includegraphics[width=1\textwidth]{OverzichtOpstelling.png}
    \caption[Overzicht van de opstelling.]{\label{fig:opstelling}Dit is de eerste illustratie die een beeld schetst van hoe de opstelling fundamenteel in elkaar zit.}
\end{figure}

\section{Testen van tunnels}
\label{sec:testen van tunnels}

Als tweede stap van het project wordt een vergelijkende studie uitgevoerd tussen twee tunnelingtechnologieën.  
De eenvoudigste methode is het opzetten van een Layer 2-tunnel tussen beide toestellen, waardoor ze functioneren alsof ze op hetzelfde netwerk zijn aangesloten.  
Een complexere opstelling maakt daarentegen gebruik van een combinatie van een Layer 2-verbinding en een VPN-tunnel.  

De vergelijking tussen deze twee technologieën richt zich voornamelijk op de verschillende MTU-groottes die vereist zijn voor een correcte gegevensoverdracht.  
Welke prestatieverschillen bestaan er tussen beide? Zijn ze allebei geschikt voor deze opstelling? En heeft een van de twee een duidelijke voorkeur?  

Bij deze analyse moet vooral rekening worden gehouden met de pakketgrootte om te voorkomen dat gegevensfragmentatie optreedt, aangezien beide technologieën de pakketstructuur vergroten.

\section{Automatisatie}
\label{sec:automatisatie}

Als laatste stap in het project wordt ervoor gezorgd dat de opstelling op een geautomatiseerde manier kan worden opgezet met behulp van een Ansible-playbook.  
Dit playbook is zo geschreven dat het verbinding maakt met de twee verschillende toestellen en de benodigde configuratie implementeert.  

Indien alles correct verloopt, resulteert dit in een snelle en eenvoudige oplossing om IPv6-connectiviteit te voorzien voor de labo’s die in het netwerklabo worden uitgevoerd.