%%=============================================================================
%% Methodologie
%%=============================================================================

\chapter{\IfLanguageName{dutch}{Methodologie}{Methodology}}%
\label{ch:methodologie}

%% TODO: In dit hoofstuk geef je een korte toelichting over hoe je te werk bent
%% gegaan. Verdeel je onderzoek in grote fasen, en licht in elke fase toe wat
%% de doelstelling was, welke deliverables daar uit gekomen zijn, en welke
%% onderzoeksmethoden je daarbij toegepast hebt. Verantwoord waarom je
%% op deze manier te werk gegaan bent.
%% 
%% Voorbeelden van zulke fasen zijn: literatuurstudie, opstellen van een
%% requirements-analyse, opstellen long-list (bij vergelijkende studie),
%% selectie van geschikte tools (bij vergelijkende studie, "short-list"),
%% opzetten testopstelling/PoC, uitvoeren testen en verzamelen
%% van resultaten, analyse van resultaten, ...
%%
%% !!!!! LET OP !!!!!
%%
%% Het is uitdrukkelijk NIET de bedoeling dat je het grootste deel van de corpus
%% van je bachelorproef in dit hoofstuk verwerkt! Dit hoofdstuk is eerder een
%% kort overzicht van je plan van aanpak.
%%
%% Maak voor elke fase (behalve het literatuuronderzoek) een NIEUW HOOFDSTUK aan
%% en geef het een gepaste titel.

\section{Voorbereiding}
\label{sec:voorbereiding}

Aangezien een groot deel van de netwerkapparatuur afkomstig is van MikroTik, zal de eerste stap bestaan uit een bijscholingstraject waarin een certificaat behaald kan worden.
Deze bijscholing kan dan gebruikt worden voor het efficiënt configureren van MikroTik apparatuur.
Na het behalen van dit certificaat wordt er een overleg gepland met de co-promotor, waarin de opdracht verder toegelicht wordt. 
Tijdens dit overleg worden de doelstellingen verduidelijkt, en wordt er bepaald welke specifieke toestellen geconfigureerd moeten worden en op welke manier dit het beste kan gebeuren. 
Vervolgens zal er uitgebreide documentatie verzameld worden die nodig is voor het schrijven van de scripts voor de verschillende toestellen. 
Deze documentatie bevat technische handleidingen, configuratievoorbeelden en richtlijnen die nodig zijn voor een correcte implementatie. 
Tot slot wordt in deze fase bepaald welke IPv6-adressen aan welke gebruikers of apparaten worden toegewezen. 
Er wordt ook een schema opgesteld voor de adressen of adressenreeksen die gereserveerd moeten blijven voor specifieke doeleinden, zoals netwerkapparatuur, 
servers of andere infrastructuurelementen.

\section{Ontwerp}
\label{sec:ontwerp}

De volgende stap is de uitwerking van het configuratiescript, welke gebeurt via een Ansible Playbook. 
Dit playbook maakt gebruik van verschillende YAML-bestanden, zodat er een duidelijke scheiding kan worden gemaakt tussen de configuraties voor de verschillende typen toestellen. 
Het Ansible Playbook zelf is een zelfgeschreven verzameling van bestanden die alle nodige configuraties en benodigdheden bevat. 
Hierbij worden de configuraties opgesplitst in afzonderlijke taken, zodat iedere stap duidelijk is. 
De ontwerpfase omvat ook het testen van het playbook in een gecontroleerde omgeving, waar de scripts kunnen worden verfijnd voordat ze in de productieomgeving worden uitgerold.

\section{Testing}
\label{sec:testing}

Het testen van het script zal plaatsvinden in een labo-omgeving, waarbij een reeks virtuele machines op de testservers zal worden ingezet. 
Deze machines dienen als een soort 'sandbox', waarin de configuraties veilig kunnen worden getest zonder de productieomgeving te beïnvloeden. 
De labo-omgeving zal ook dienen als proof of concept, waarbij de functionaliteit en betrouwbaarheid van het playbook grondig worden getest.
In deze fase worden de configuraties getest, en worden eventuele problemen of fouten genoteerd. 
Ook wordt hier gekeken of alles functioneert zoals verwacht en kan er gekeken worden welke toestellen op welke manier compatibel zijn. 
Bij elk probleem wordt onderzocht hoe dit kan worden opgelost, en wordt gekeken of bepaalde delen van het script kunnen worden verbeterd of vereenvoudigd. 
Indien er grote delen van de configuratie kunnen worden samengevoegd of aangepast, wordt dit gedaan om een robuust en efficiënt playbook te ontwikkelen.
Het einddoel van de testfase is om een playbook te creëren dat betrouwbaar en flexibel genoeg is om gebruikt te worden op verschillende typen netwerkapparatuur, 
met de garantie dat het script effectief en zonder grote complicaties werkt.

\section{Deployment}
\label{sec:deployment}

Na het testen van het playbook kan er een rapport worden opgesteld, waarin de opgetreden problemen, fouten en de daaropvolgende oplossingen worden gedocumenteerd. 
Dit rapport zal ook aanbevelingen bevatten voor eventuele toekomstige verbeteringen of uitbreidingen van het playbook. 
Bovendien zal er al een eerste versie van het playbook beschikbaar zijn, die klaar is voor implementatie in de productieomgeving.
Indien nodig kan het playbook verder worden uitgewerkt en geoptimaliseerd door SmartEye, om te zorgen dat het geschikt is voor de grotere netwerken van de organisatie. 
Het einddoel van deze fase is om een goed werkend en geoptimaliseerd playbook te leveren dat kan worden toegepast op de netwerkapparatuur van de organisatie. 
Het playbook zal dan verder worden ingezet in de bredere netwerkinfrastructuur, wat de netwerkbeheerprocessen zal automatiseren en vereenvoudigen.


