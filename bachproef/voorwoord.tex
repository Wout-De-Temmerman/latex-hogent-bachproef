%%=============================================================================
%% Voorwoord
%%=============================================================================

\chapter*{\IfLanguageName{dutch}{Woord vooraf}{Preface}}%
\label{ch:voorwoord}

%% TODO:
%% Het voorwoord is het enige deel van de bachelorproef waar je vanuit je
%% eigen standpunt (``ik-vorm'') mag schrijven. Je kan hier bv. motiveren
%% waarom jij het onderwerp wil bespreken.
%% Vergeet ook niet te bedanken wie je geholpen/gesteund/... heeft

Al van jongs af aan ben ik gefascineerd door de werking van netwerken en de essentiële rol die ze spelen in onze digitale wereld.  
Tijdens mijn opleiding Toegepaste Informatica aan HOGENT werd mijn interesse hierin verder aangewakkerd.  
Een uitspraak van mijn lector Netwerken is me daarbij altijd bijgebleven:  

\begin{quote}
    ``Het is niet logisch dat onze school IPv6, een technologie die al geruime tijd bestaat, bespreekt in de netwerklessen, maar er zelf nog geen gebruik van maakt.''
\end{quote}

Dit onderzoek is daarom opgesteld om deze situatie op een eenvoudige manier te omzeilen en een duidelijk onderscheid te maken tussen de verschillende tunneltechnieken die hiermee gepaard gaan.\\

Bij deze wil ik graag enkele personen bedanken die een essentiële rol hebben gespeeld bij deze bachelorproef.  
Allereerst wil ik mijn co-promotor en lector, Andy Van Maele, bedanken voor zijn begeleiding, advies en waardevolle feedback gedurende dit project.  
Daarnaast ben ik mijn promotor, Martijn Saelens, dankbaar voor zijn interesse en bereidheid om op het laatste moment nog de rol van promotor op zich te nemen.
Ook wil ik lector Bert Van Vreckem bedanken voor zijn hulp bij de opzet van de virtuele machines die nodig waren voor de proof of concept.
Tot slot wil ik ook mijn bachelorproefcoördinator, Lena De Mol, bedanken voor haar ondersteuning en hulp bij het bewaken van het verloop van de proef.  
